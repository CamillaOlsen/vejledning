\documentclass[a4paper,11pt]{article}

\usepackage{revy}
\usepackage[latin1]{inputenc}
\usepackage[T1]{fontenc}
\usepackage[danish]{babel}

\revyname{FysikRevy}
\revyyear{2012}
% HUSK AT OPDATERE VERSIONSNUMMER
\version{1.0}
\eta{$3$ minutter}
% Her skrives et estimat af sangens/sketchens varighed
\status{f�rdig/ideer}
% skriv \status{f�rdig} hvis sketchen er f�rdig, ellers \status{ideer}

\title{Sketchens navn}
\author{en forfatter}

\begin{document}
\maketitle

\begin{roles}
\role{R1}[Skuespiller 1] Rolle 1
\role{R2}[Skuespiller 2] Rolle 2
\role{MD}[Bjarke] Mogens Dam
\role{H}[Gorm] Holger
\end{roles}
%Liste over roller og deres indehavere.

\begin{props}
\prop{Briks}[Person, der skaffer]
\prop{2 lagener}[Person, der skaffer]
\end{props}
%Liste over rekvisitter. Behold teksten [Person, der skaffer],
%indtil det er sikkert, hvem der skal have ansvaret for rekvisitten

\begin{sketch}

\scene Lys op.
%Brug \scene inden alt scenespil inkl. lys og lyd fra TeXnikken.

\says{MD} Her p� scenen er jeg Mogens Dam, men til dagligt er jeg Bjarke - ussel studerende.
%\says laver en ny replik. {MD} angiver hvilken rolle, replikken tilh�rer.

\says{H}[I skinger falset] Hvor sjovt! Jeg spiller Holger p� scenen lige nu, men i virkeligheden er jeg Gorm.
%Beskrivelse af replikken i firkantet parantes

\says{MD} Det var da m�rkeligt. \act{Kl�r sig i sk�gget}
%Brug \act om skuespilsting.

\scene Lys ned

\end{sketch}
\end{document}